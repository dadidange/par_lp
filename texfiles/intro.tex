\section{Introduction}

The goal of \texttt{par} is to read a number of FASTA-sequences and to return their multiple alignment in MAF-Format\footnote{\url{http://genome.ucsc.edu/FAQ/FAQformat\#format5}}.
The approach for calculating the alignment is based on the program phylonium \citep{phylonium}. 

In \textit{andi} \citep{andi} and its faster successor \textit{phylonium} \citep{phylonium} the authors presented the idea of \textit{anchor distances} to estimate evolutionary distances. 
Anchor distances are based on micro-alignments of regions that are anchored by exact matches. 
Both approaches proved to be orders of magnitudes faster than classical alignment tools while still being accurate on closely related genomes.
Even among similar alignment-free methods they were found to be among the fastest. 
However, the actual sequence alignments that are used to generate the distance matrices stay implicit and cannot be accessed. 
The goal of this thesis is to make the alignments accessible and to provide them in a way that makes them comparable to classical alignment tools. 

The section \ref{sec:implementation} explains the main implementation of the program.
In section \ref{sec:seqUtil} the [[seqUtil]] package is explained and implemented in more detail although some parts will already be shown before.
This package contains some helpers that are in some sense related to sequences.

For andi \citep{andi} and phylonium \citep{phylonium} anchor alignments are build to estimate the distance between genomes.
For [[par]] we take these anchor alignments and show the actual underlying explicit multiple sequence alignment (MSA).
